%-----------------------------------------------------
\begin{alerttextbox}{Introductory Note}
This document is an adaption of the original datacamp.org cheat sheet.\\
\begin{itemize}
    \item {https://www.datacamp.com/resources/cheat-sheets/importing-data-python-cheat-sheet}
    \item {https://github.com/f616/Python-Importing-Data-Cheat-Sheet}
\end{itemize}

\end{alerttextbox}


%-----------------------------------------------------
\section{Importing Data in Python}

\begin{myblock}{}
\textbf{Most of the time, you’ll use either NumPy or pandas to import your data:}
\begin{codebox}{python}{}
import pandas as pd
\end{codebox}
\end{myblock}


%--------------------------------------------------------------
\section{Help}

\begin{codebox}{python}{}
np.info(np.ndarray.dtype)
help(pd.read_csv)
\end{codebox}


%--------------------------------------------------------------
\section{Text Files}

\begin{myblock}{Plain Text Files}
\begin{codebox}{python}{}
filename = 'huck_finn.txt'
file = open(filename, mode='r')  #Open the file for reading
text = file.read()  #Read a file’s contents
print(file.closed)  #Check whether file is closed
file.close()  #Close file
print(text)
\end{codebox}
\textbf{Using the context manager with}
\begin{codebox}{python}{}
with open ('huck_finn.txt', 'r') as file:
    print(file.readline())  #Read a single line
    print(file.readline())
    print(file.readline())
\end{codebox}
\end{myblock}

\begin{myblock}{Table Data: Flat Files}
\begin{myblock}{Importing Flat Files with NumPy}
\begin{codebox}{python}{}
filename = 'huck_finn.txt'
file = open(filename, mode='r')  #Open the file for reading
text = file.read()  #Read a file’s contents
print(file.closed)  #Check whether file is closed
file.close()  #Close file
print(text)
\end{codebox}
\textbf{Files with one data type}
\begin{codebox}{python}{}
filename = 'mnist.txt'
data = np.loadtxt(filename, \
    delimiter=',', \  #String used to separate values
    skiprows=2, \  #Skip the first 2 lines
    usecols=[0,2], \  #Read the 1st and 3rd column
    dtype=str)  #The type of the resulting array
\end{codebox}
\textbf{Files with mixed data type}
\begin{codebox}{python}{}
filename = 'titanic.csv'
data = np.genfromtxt(filename, \
    delimiter=',', \
    names=True, \  #Look for column header
    dtype=None)
data_array = np.recfromcsv(filename)
#The default dtype of the np.recfromcsv() function is None
\end{codebox}
\end{myblock}
\begin{myblock}{Importing Flat Files with Pandas}
\begin{codebox}{python}{}
filename = 'winequality-red.csv'
data = pd.read_csv(filename, \
    nrows=5, \  #Number of rows of file to read
    header=None, \  #Row number to use as col names
    sep='\t', \  #Delimiter to use
    comment='#', \  #Character to split comments
    na_values=[""])  #String to recognize as NA/NaN
\end{codebox}
\end{myblock}
\end{myblock}


%--------------------------------------------------------------
\section{Exploring Your Data}

\begin{codebox}{python}{NumPy Arrays}
data_array.dtype  #Data type of array elements
data_array.shape  #Array dimensions
len(data_array)  #Length of array
\end{codebox}

\begin{codebox}{python}{Pandas DataFrames}
df.head()  #Return first DataFrame rows
df.tail()  #Return last DataFrame rows
df.index  #Describe index
df.columns  #Describe DataFrame columns
df.info()  #Info on DataFrame
data_array = data.values  #Convert a DataFrame to an a NumPy array
\end{codebox}

%--------------------------------------------------------------
\section{SAS File}

\begin{codebox}{python}{}
from sas7bdat import SAS7BDAT
with SAS7BDAT('urbanpop.sas7bdat') as file:
    df_sas = file.to_data_frame()
\end{codebox}


%--------------------------------------------------------------
\section{Stata File}

\begin{codebox}{python}{}
data = pd.read_stata('urbanpop.dta')
\end{codebox}


%--------------------------------------------------------------
\section{Excel Spreadsheets}

\begin{myblock}{}
\begin{codebox}{python}{}
file = 'urbanpop.xlsx'
data = pd.ExcelFile(file)
df_sheet2 = data.parse('1960-1966', \
                    skiprows=[0], \
                    names=['Country', \
                    'AAM: War(2002)'])
df_sheet1 = data.parse(0, \
                    parse_cols=[0], \
                    skiprows=[0], \
                    names=['Country'])
\end{codebox}
\textbf{To access the sheet names{,} use the }sheet\_names \textbf{ attribute:}
\begin{codebox}{python}{}
data.sheet_names
\end{codebox}
\end{myblock}

%--------------------------------------------------------------
\section{Relational Databases}

\begin{myblock}{}
\begin{myblock}{}
\begin{codebox}{python}{}
from sqlalchemy import create_engine
engine = create_engine('sqlite://Northwind.sqlite')
\end{codebox}
\textbf{Use the }table\_names() \textbf{ method to fetch a list of table names:}
\begin{codebox}{python}{}
table_names = engine.table_names()
\end{codebox}
\end{myblock}

\begin{myblock}{Querying Relational Databases}
\begin{codebox}{python}{}
con = engine.connect()
rs = con.execute("SELECT * FROM Orders")
df = pd.DataFrame(rs.fetchall())
df.columns = rs.keys()
con.close()
\end{codebox}
\textbf{Using the context manager with}
\begin{codebox}{python}{}
with engine.connect() as con:
    rs = con.execute("SELECT OrderID FROM Orders")
    df = pd.DataFrame(rs.fetchmany(size=5))
    df.columns = rs.keys()
\end{codebox}
\end{myblock}

\begin{myblock}{Querying Relational Databases with Pandas}
\begin{codebox}{python}{}
df = pd.read_sql_query("SELECT * FROM Orders", engine)
\end{codebox}
\end{myblock}
\end{myblock}


%--------------------------------------------------------------
\section{Pickled Files}

\begin{codebox}{python}{}
import pickle
with open('pickled_fruit.pkl', 'rb') as file:
    pickled_data = pickle.load(file)
\end{codebox}


%--------------------------------------------------------------
\section{Matlab Files}

\begin{codebox}{python}{}
import scipy.io
filename = 'workspace.mat'
mat = scipy.io.loadmat(filename)
\end{codebox}


%--------------------------------------------------------------
\section{HDF5 Files}

\begin{codebox}{python}{}
import h5py
filename = 'H-H1_LOSC_4_v1-815411200-4096.hdf5'
data = h5py.File(filename, 'r')
\end{codebox}


%--------------------------------------------------------------
\section{Exploring Dictionaries}

\begin{codebox}{python}{Querying relational databases with pandas}
print(mat.keys())  #Print dictionary keys
for key in data.keys():  #Print dictionary keys
    print(key)

meta
quality
strain

pickled_data.values()  #Return dictionary values
print(mat.items())  #Returns items in list format of (key, value) tuple pairs
\end{codebox}

\begin{codebox}{python}{Accessing Data Items with Keys}
for key in data ['meta'].keys():  #Explore the HDF5 structure
    print(key)

Description
DescriptionURL
Detector
Duration
GPSstart
Observatory
Type
UTCstart

print(data['meta']['Description'].value)  #Retrieve the value for a key
\end{codebox}

%--------------------------------------------------------------
\section{Navigating Your FileSystem}

\begin{codebox}{python}{Magic Commands}
!ls  #List directory contents of files and directories
%cd ..  #Change current working directory
%pwd  #Return the current working directory path
\end{codebox}

\begin{codebox}{python}{OS Library}
import os
path = '/usr/tmp'
wd = os.getcwd()  #Store the name of current directory in a string
os.listdir(wd)  #Output contents of the directory in a list
os.chdir(path)  #Change current working directory
os.rename('test1.txt', 'test2.txt')  #Rename a file
os.remove('test1.txt')  #Delete an existing file
os.mkdir('newdir')  #Create a new directory
\end{codebox}
